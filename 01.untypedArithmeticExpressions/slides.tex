\documentclass{beamer}

\usepackage{busproofs}

\usepackage{amsmath}

\usefonttheme[onlymath]{serif}

\usepackage{ifxetex,ifluatex}
\usepackage{etoolbox}
% \usepackage[svgnames]{xcolor}

\usepackage{tikz}

\usepackage{framed}

% conditional for xetex or luatex
\newif\ifxetexorluatex
\ifxetex
  \xetexorluatextrue
\else
  \ifluatex
    \xetexorluatextrue
  \else
    \xetexorluatexfalse
  \fi
\fi
%
\ifxetexorluatex%
  \usepackage{fontspec}
  \usepackage{libertine} % or use \setmainfont to choose any font on your system
  \newfontfamily\quotefont[Ligatures=TeX]{Linux Libertine O} % selects Libertine as the quote font
\else
  \usepackage[utf8]{inputenc}
  \usepackage[T1]{fontenc}
  \usepackage{libertine} % or any other font package
  \newcommand*\quotefont{\fontfamily{LinuxLibertineT-LF}} % selects Libertine as the quote font
\fi

\newcommand*\quotesize{60} % if quote size changes, need a way to make shifts relative
% Make commands for the quotes
\newcommand*{\openquote}
   {\tikz[remember picture,overlay,xshift=-4ex,yshift=-2.5ex]
   \node (OQ) {\quotefont\fontsize{\quotesize}{\quotesize}\selectfont``};\kern0pt}

\newcommand*{\closequote}[1]
  {\tikz[remember picture,overlay,xshift=4ex,yshift={#1}]
   \node (CQ) {\quotefont\fontsize{\quotesize}{\quotesize}\selectfont''};}

% select a colour for the shading
\colorlet{shadecolor}{lightgray}

\newcommand*\shadedauthorformat{\emph} % define format for the author argument

% Now a command to allow left, right and centre alignment of the author
\newcommand*\authoralign[1]{%
  \if#1l
    \def\authorfill{}\def\quotefill{\hfill}
  \else
    \if#1r
      \def\authorfill{\hfill}\def\quotefill{}
    \else
      \if#1c
        \gdef\authorfill{\hfill}\def\quotefill{\hfill}
      \else\typeout{Invalid option}
      \fi
    \fi
  \fi}
% wrap everything in its own environment which takes one argument (author) and one optional argument
% specifying the alignment [l, r or c]
%
\newenvironment{shadequote}[2][l]%
{\authoralign{#1}
\ifblank{#2}
   {\def\shadequoteauthor{}\def\yshift{-2ex}\def\quotefill{\hfill}}
   {\def\shadequoteauthor{\par\authorfill\shadedauthorformat{#2}}\def\yshift{2ex}}
\begin{snugshade}\begin{quote}\openquote}
{\shadequoteauthor\quotefill\closequote{\yshift}\end{quote}\end{snugshade}}


\title{Types and Programming Languages} 
\subtitle{Untyped Arithmetic Expressions}

\author{Rodrigo Bonif\'{a}cio}
\date{2017/08}

\begin{document}

\begin{frame}
\titlepage
\end{frame}

\begin{frame}
\begin{shadequote}[l]{Benjamin Pierce}
\ldots a small language of numbers and booleans'' \ldots though 
``a straightfoward vehicle for the introduction of several fundamental concepts
\end{shadequote}

\pause 
\begin{itemize}
\item abstract syntax tree
\item inductive definitions 
\item proofs
\end{itemize}

\end{frame}

\begin{frame}
\frametitle{Syntactic Forms} 

\begin{itemize}
\item boolean constants: \texttt{true}, \texttt{false}
\item conditional expressions
\item numeric constant: \texttt{zero}
\item arithmetic operators: \texttt{succ}, \texttt{pred}
\item testins operation: \texttt{isZero} 
\end{itemize}
\end{frame}

\begin{frame}[fragile]
\frametitle{Grammar} 

\begin{verbatim}
 Exp ::= true
         false 
         if Exp then Exp else Exp
         0
         succ Exp
         pred Exp
         iszero Exp
\end{verbatim}
\end{frame}

\begin{frame}
In this language, the results of evaluation 
are terms of a particularly simple form: they will 
allways be either boolean constants or numbers (values). 
\end{frame}

\begin{frame}
The (BNF like) grammar is just a compact notation 
for the following inductive definition (often found 
in Computer Science Logic Books). 

\begin{definition}{{\color{blue}Expressions, inductively.}}
The set of expressions is the smallest set $T$
such that

\begin{enumerate}
  \item $\{true, false, 0\}\subseteq\ T$
  \item if $t_1 \in\ T$ then $\{succ\ t_1, pred\ t_1, iszero\ t_1\} \subseteq\ T$
  \item if $t_1 \in\ T$, $t_2 \in\ T$, and  $t_3 \in\ T$, then $if\ t_1\ then\ t_2\ else\ t_3 \in\ T$
\end{enumerate}
\end{definition} 

\pause The book presents two other alternatives for specifying the 
syntax of a language. \pause Here we will mostly work with BNF 
based grammar specification. 

\end{frame}

\begin{frame}
\frametitle{Induction on terms} 

The inductive structure of expressions allow us 
to give inductive definitions of functions over 
the set of expressions. 

\begin{definition}{{\color{blue}The set of constants of an expression}}
\begin{eqnarray*}
Consts(true)         & = & \{true\}         \\
Consts(false)        & = & \{false\}        \\
Consts(0)            & = & {0}            \\ \pause 
Consts(succ\ t_1)    & = & Consts(t_1)    \\ 
Consts(pred\ t_1)    & = & Consts(t_1)    \\ 
Consts(iszero\ t_1)  & = & Consts(t_1)    \\ \pause 
Consts(if\ t_1 then\ t_2 else t_3) & = & Consts(t_1) \cup Consts(t_2) \cup Consts(t_3) \\  
\end{eqnarray*}

\end{definition} 
\end{frame}

\begin{frame}
\frametitle{Small step operational semantics}


\begin{shadequote}[l]{Benjamin Pierce}
\ldots An \emph{evaluation relation} on expressions, written 
as $t \rightarrow t'$ and pronounced ``$t$ evaluates to $t'$ in 
one step''. The intuition is that, if $t$ is the state of a 
computation at a given moment, then the computation can make 
a step and change its state to $t'$. The evaluation relation 
is defined using a set of inference rules (axioms and proper rules). 
\end{shadequote}

\end{frame}


\begin{frame}
\begin{definition}{Normal Form}
A term $t$ is in normal form if no evaluation rule 
applies to it---i.e., if there is no $t'$ such that 
$t \rightarrow t'$. \pause Every value is in normal 
form. \pause Normal forms that are not values represent 
that something went wrong.  
\end{definition}
\end{frame}
\begin{frame}
\frametitle{Axioms (computation rules)} 

\begin{prooftree}
\AxiomC{}
\RightLabel{\quad (E-PredZero)}
\UnaryInfC{$pred\ 0 \rightarrow 0$}
\end{prooftree} 

\begin{prooftree}
\AxiomC{}
\RightLabel{\quad (E-PredSucc)}
\UnaryInfC{$pred(succ\ t_1) \rightarrow t_1$}
\end{prooftree} 


\begin{prooftree}
\AxiomC{}
\RightLabel{\quad (E-IsZeroZero)}
\UnaryInfC{$iszero\ 0 \rightarrow true$}
\end{prooftree} 

\begin{prooftree}
\AxiomC{}
\RightLabel{\quad (E-IsZeroSucc)}
\UnaryInfC{$iszero(succ\ t_1) \rightarrow false$}
\end{prooftree} 

\begin{prooftree}
\AxiomC{}
\RightLabel{\quad (E-IfTrue)}
\UnaryInfC{$if\ true\ then\ t_1\ else\ t_2 \rightarrow t_1$} 
\end{prooftree}

\begin{prooftree}
\AxiomC{}
\RightLabel{\quad (E-IfFalse)}
\UnaryInfC{$if\ false\ then\ t_1\ else\ t_2 \rightarrow t_2$}
\end{prooftree}

\end{frame}

\begin{frame}
\frametitle{Proper rules (congruence rules)} 

\begin{prooftree}
\AxiomC{$t_1 \rightarrow t_1'$}
\RightLabel{\quad (E-If)}
\UnaryInfC{$if\ t_1\ then\ t_2\ else\ t_3 \rightarrow if\ t_1' \ then\ t_2\ else\ t_3$}
\end{prooftree}

\begin{prooftree}
\AxiomC{$t_1 \rightarrow t_1'$}
\RightLabel{\quad (E-Succ)}
\UnaryInfC{$succ\ t_1\ \rightarrow succ\ t_1'$}
\end{prooftree}

\begin{prooftree}
\AxiomC{$t_1 \rightarrow t_1'$}
\RightLabel{\quad (E-Pred)}
\UnaryInfC{$pred\ t_1\ \rightarrow pred\ t_1'$}
\end{prooftree}

\begin{prooftree}
\AxiomC{$t_1 \rightarrow t_1'$}
\RightLabel{\quad (E-IsZero)}
\UnaryInfC{$iszero\ t_1\ \rightarrow iszero\ t_1'$}
\end{prooftree}
\end{frame}


\begin{frame}
Let's code the Untyped Arithmetic Expressions in Haskell \ldots
\end{frame}
\end{document}
