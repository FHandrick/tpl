\documentclass{beamer}

\usepackage{syntax}

\usepackage{busproofs}

\usepackage{amsmath}

\usefonttheme[onlymath]{serif}

\usepackage{ifxetex,ifluatex}
\usepackage{etoolbox}
% \usepackage[svgnames]{xcolor}

\usepackage{tikz}

\usepackage{framed}

% conditional for xetex or luatex
\newif\ifxetexorluatex
\ifxetex
  \xetexorluatextrue
\else
  \ifluatex
    \xetexorluatextrue
  \else
    \xetexorluatexfalse
  \fi
\fi
%
\ifxetexorluatex%
  \usepackage{fontspec}
  \usepackage{libertine} % or use \setmainfont to choose any font on your system
  \newfontfamily\quotefont[Ligatures=TeX]{Linux Libertine O} % selects Libertine as the quote font
\else
  \usepackage[utf8]{inputenc}
  \usepackage[T1]{fontenc}
  \usepackage{libertine} % or any other font package
  \newcommand*\quotefont{\fontfamily{LinuxLibertineT-LF}} % selects Libertine as the quote font
\fi

\newcommand*\quotesize{60} % if quote size changes, need a way to make shifts relative
% Make commands for the quotes
\newcommand*{\openquote}
   {\tikz[remember picture,overlay,xshift=-4ex,yshift=-2.5ex]
   \node (OQ) {\quotefont\fontsize{\quotesize}{\quotesize}\selectfont``};\kern0pt}

\newcommand*{\closequote}[1]
  {\tikz[remember picture,overlay,xshift=4ex,yshift={#1}]
   \node (CQ) {\quotefont\fontsize{\quotesize}{\quotesize}\selectfont''};}

% select a colour for the shading
\colorlet{shadecolor}{lightgray}

\newcommand*\shadedauthorformat{\emph} % define format for the author argument

% Now a command to allow left, right and centre alignment of the author
\newcommand*\authoralign[1]{%
  \if#1l
    \def\authorfill{}\def\quotefill{\hfill}
  \else
    \if#1r
      \def\authorfill{\hfill}\def\quotefill{}
    \else
      \if#1c
        \gdef\authorfill{\hfill}\def\quotefill{\hfill}
      \else\typeout{Invalid option}
      \fi
    \fi
  \fi}
% wrap everything in its own environment which takes one argument (author) and one optional argument
% specifying the alignment [l, r or c]
%
\newenvironment{shadequote}[2][l]%
{\authoralign{#1}
\ifblank{#2}
   {\def\shadequoteauthor{}\def\yshift{-2ex}\def\quotefill{\hfill}}
   {\def\shadequoteauthor{\par\authorfill\shadedauthorformat{#2}}\def\yshift{2ex}}
\begin{snugshade}\begin{quote}\openquote}
{\shadequoteauthor\quotefill\closequote{\yshift}\end{quote}\end{snugshade}}



\title{Types and Programming Languages} 
\subtitle{Untyped Lambda Calculus}

\author{Rodrigo Bonif\'{a}cio}

\begin{document}

\setlength{\grammarparsep}{20pt plus 1pt minus 1pt} 
\setlength{\grammarindent}{4em}

\begin{frame}
\titlepage
\end{frame}

\begin{frame}
  \frametitle{$Lambda-Calculus$} 
\begin{shadequote}[l]{Benjamin Pierce}
  \ldots a formal system in which all computation
  is reduced to the basic operations of function definition
  and application. 
\end{shadequote}\pause

\begin{shadequote}[l]{Benjamin Pierce}
  \ldots Its importance arises from
  the fact that it can be viewed simultaneously as a {\color{blue}simple
  programming language and as a mathematical object} about
  which rigorous statements can be proved.
\end{shadequote}

\end{frame}

\begin{frame}

  \begin{shadequote}[l]{Benjamin Pierce}
    In the $lambda-calculus$ everything is a function: the arguments
    accepted by functions are themselves functions and the result returned
    by a function is another function. 
  \end{shadequote} \pause

  \begin{block}{BNF Grammar}
    \begin{grammar}
      <Exp> ::= x
      \alt      $\lambda$x.<Exp>
      \alt      <Exp> <Exp> 
    \end{grammar}  
  \end{block}
\end{frame}

\begin{frame}
  \frametitle{Syntax}

  Let $V$ be a countable set of variable names. The
  set of terms is the smallest set $T$ such that

  \begin{enumerate}
    \item $x \in T$ for every $x \in V$
    \item if $t_1 \in T$ and $x \in V$, then $\lambda x.t_1 \in T$
    \item if $t_1 \in T$ and $t_2 in T$, then $t_1\ t_2 \in T$   
  \end{enumerate}

  \pause
  The set of \emph{free variables} of a term $t$, written as
  $FV(t)$, is defined as follows:

  \begin{eqnarray*}
    FV(x) & = & {x} \\
    FV(\lambda x . t_1) & = &  FV(t_1) \backslash {x} \\
    FV(t_1\ t_2) & = & FV(t_1) \cup FV(t_2) 
  \end{eqnarray*}
  
\end{frame}

\begin{frame}
  \frametitle{Operational Semantics}

  \begin{shadequote}[l]{Benjamin Pierce}
    The computation consists of rewriting
    an application whose left-hand component is an
    abstraction, by {\color{blue}substituting} the right-hand
    component for the bound variable in the abstraction's body. 
  \end{shadequote} \pause

  \begin{block}{Graphically}
       \begin{prooftree}
       \AxiomC{$(\lambda x . e) e_2 \rightarrow [x \mapsto e_2]e$}
       \end{prooftree}
  \end{block}

  \pause
  \begin{itemize}
  \item where, $[x \mapsto e_2]e$ means ``the term obtained by
    replacing all \emph{free occurrences} of $x$ in $e$ by
    $e_2$''. 
  \end{itemize}
\end{frame}

\begin{frame}
       \begin{prooftree}
         \AxiomC{$t_1 \rightarrow t_1'$}
         \RightLabel{\quad (E-App1)}
         \UnaryInfC{$t_1\ t_2 \rightarrow t_1'\ t_2$}
       \end{prooftree}
       
       \begin{prooftree}
         \AxiomC{$t_2 \rightarrow t_2'$}
         \RightLabel{\quad (E-App2)}
         \UnaryInfC{$t_1\ t_2 \rightarrow t_1\ t_2'$}
       \end{prooftree}

       \begin{prooftree}
         \AxiomC{}
         \RightLabel{\quad (E-AppAbs)}
         \UnaryInfC{$(\lambda x . t_{12}) v_2 \rightarrow [x \mapsto v_2]t_{12}$}
       \end{prooftree}

\end{frame}

\begin{frame}[fragile]
\frametitle{Some examples of interesting functions}

 \begin{description}
    \item[Identity:] $(\lambda x . x)$ 
    \item[Function application:] $(\lambda f . \lambda a . (f a)$
    \item[Self application:] $(\lambda s . s s)$        
  \end{description}
  
\end{frame}

\begin{frame}
  However, let's follow a seamless
  transition from \emph{Untyped Arithmetic Expression}
  to the \emph{Untyped Lambda Calculus}\pause, discussing
  an intermediate (and more complex) language: \textsc{RBL-Calculus} 
\end{frame}

\begin{frame}
  \frametitle{\textsc{RBL-Calculus} syntax}

\begin{grammar}
  <Exp> ::= True
  \alt False
  \alt Num
  \alt Add <Exp> <Exp> \alt Sub <Exp> <Exp> 
  \alt And <Exp> <Exp> \alt Or <Exp> <Exp> \alt Not <Exp>
  \alt Let Var <Exp> <Exp>
  \alt Var 
  \alt $\lambda$ Var  <Exp>
  \alt <Exp> <Exp> 
\end{grammar}
\end{frame}

\begin{frame}[fragile]
  \frametitle{Some examples of programs in \textsc{RBL-Calculus}}

\begin{verbatim}
  $ add 5 3 
  $ add 5 (5 3) 
  $ let x = 5 in x + x
  $ let inc = \x . x + 1 in inc 4   
\end{verbatim} 
\end{frame}

\begin{frame}
  \frametitle{Scope and Substitution}

  \texttt{let x = $e_1$ in $e_2$} \pause
  
  \texttt{let x = 5 in x + x} \pause 

  \texttt{let x = 5 in let x = 3 in x + 3} \pause  

  \texttt{let x = 5 in x + let y = x in x + y} 

\end{frame}

\begin{frame}
\huge{a bit of Haskell, \ldots + some test cases} 
\end{frame}

\begin{frame}[fragile]
  How can we translate \textsc{RBL-Calculus} programs into
  \textsc{$\lambda-calculus$}?

  \pause

  \begin{block}{We will need a set of Haskell modules}
    \begin{description}
      \item[RBL:] the \textsc{RBL-Calculus} implementation
      \item[Lambda:] the \textsc{$\lambda-calculus$} implementation
      \item[Compiler:] the translator implementation
    \end{description}
  \end{block}

\begin{center}
\begin{verbatim}
translate :: R.Expression -> L.Term
\end{verbatim}
\end{center}
\end{frame}


\begin{frame}
  \frametitle{How to translate \textsc{If-Then-Else} expressions into $\lambda$-terms?} 

  \begin{itemize}
    \item that seems easy. it is just a \texttt{test} function that 
    expects three arguments: if the first is \texttt{true} 
    this function returns the second argument; otherwise, 
    it returns the third.  \pause 

    \item in this way, we can envision that \texttt{true} is a 
    function that returns the first of a pair and \texttt{false} is 
    a function that returns the second of a pair.  \pause 
  \end{itemize} 

  \begin{center}
    \begin{eqnarray*}
      true  & = & \lambda x . \lambda y . x \\
      false & = & \lambda x . \lambda y . y  \\
      test  & = & \lambda c . \lambda e_1 . \lambda e_2 . c\ e_1\ e_2 
    \end{eqnarray*}
  \end{center}

\end{frame}

\begin{frame}
 \frametitle{How to translate \textsc{Let} expressions into $\lambda$-terms?} 

 First, let's consider the simple expression in \textsc{RBL} $Let x = 10\ in\ x + x$. \pause 
 We can rewrite this expression in terms of an \textsc{RBL} lambda expression\pause, that is 
 $(\lambda x . x + x) (10)$. Therefore, \texttt{Let} expressions are redundant even in 
 the \textsc{RBL-Calculus}.
  
\end{frame}

\begin{frame}
 \frametitle{How to translate numbers into $\lambda$ terms?}

  First of all, we can represent natural numbers using the 
  following encoding   
  
  \begin{eqnarray*}
   C_0 & = & \lambda s . \lambda z . z \\ 
   C_1 & = & \lambda s . \lambda z . s\ z \\ 
   C_2 & = & \lambda s . \lambda z . s (s\ z) \\  
   \ldots & = & \ldots
  \end{eqnarray*}

  \pause That is beautiful. Any number $n$ is an \emph{activity} 
  entity, which \emph{does something ``n'' times}. 
  We can define the successor function on \emph{Church numerals} as follows: 
  
  \begin{eqnarray*} 
    succ = \lambda n . \lambda s . \lambda z . s (n\ s\ z) \\ \pause 
   \end{eqnarray*} 
\end{frame}

\begin{frame}
\frametitle{Addition (plus)} 

 \begin{eqnarray*} 
   plus\ c_3\ c_2 & = & plus\ {\color{blue}(\lambda s.\lambda z. s\ (s\ (s\ zero)))}\ 
                              {\color{red}(\lambda s. \lambda z. s\ (s\ zero))} \\  \pause
                  & = & \lambda s. \lambda z. {\color{blue}s\ (s\ (s} {\color{red}\ (s\ (s\ zero))} \\ \pause 
   plus & = & \lambda m . \lambda n . \lambda s . \lambda z . m\ s\ (n\ s\ z) 
 \end{eqnarray*} 

\end{frame}

\begin{frame}
 \frametitle{Multiplication (times)} 

 \begin{eqnarray*} 
   times\ c_3\ c_2 & = & times (\lambda s . \lambda z .\ s\ (s\ (s\ zero)))\ 
                               (\lambda s . \lambda z .\ s\ (s\ (s\ zero))) \\ \pause
                   & = & \lambda s . \lambda z .\ s\ (s\ (s\ (s\ (s\ (s\ (s\ (s\ (s\ zero)))))))) \\ \pause  
   times & = & \lambda m . \lambda n . \lambda s . \lambda z . m\ (plus\ n)\ c_0 \pause
 \end{eqnarray*} 

  That is, $times\ c_3\ c_2$ yields $plus\ c_2\ (plus\ c_2\ (plus\ c_2\ zero))$.   \pause 

  \begin{eqnarray*}
   iszero & = & \lambda m . m (\lambda x . false) true 
  \end{eqnarray*} 
 
\end{frame}

\begin{frame}
\frametitle{Other useful combinators}

\begin{eqnarray*}
 omega & = & (\lambda x . x)\ (\lambda x . x) \\
 fix & = & \lambda f . (\lambda x . f (\lambda y . x\ x\ y))\ (\lambda x . f (\lambda y . x\ x\ y))
\end{eqnarray*}
\end{frame}

\begin{frame}
\titlepage
\end{frame}
\end{document}
