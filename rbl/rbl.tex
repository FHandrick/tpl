\documentclass{beamer}

\usepackage{syntax}

\usepackage{busproofs}

\usepackage{amsmath}

\usefonttheme[onlymath]{serif}

\usepackage{ifxetex,ifluatex}
\usepackage{etoolbox}
% \usepackage[svgnames]{xcolor}

\usepackage{tikz}

\usepackage{framed}

% conditional for xetex or luatex
\newif\ifxetexorluatex
\ifxetex
  \xetexorluatextrue
\else
  \ifluatex
    \xetexorluatextrue
  \else
    \xetexorluatexfalse
  \fi
\fi
%
\ifxetexorluatex%
  \usepackage{fontspec}
  \usepackage{libertine} % or use \setmainfont to choose any font on your system
  \newfontfamily\quotefont[Ligatures=TeX]{Linux Libertine O} % selects Libertine as the quote font
\else
  \usepackage[utf8]{inputenc}
  \usepackage[T1]{fontenc}
  \usepackage{libertine} % or any other font package
  \newcommand*\quotefont{\fontfamily{LinuxLibertineT-LF}} % selects Libertine as the quote font
\fi

\newcommand*\quotesize{60} % if quote size changes, need a way to make shifts relative
% Make commands for the quotes
\newcommand*{\openquote}
   {\tikz[remember picture,overlay,xshift=-4ex,yshift=-2.5ex]
   \node (OQ) {\quotefont\fontsize{\quotesize}{\quotesize}\selectfont``};\kern0pt}

\newcommand*{\closequote}[1]
  {\tikz[remember picture,overlay,xshift=4ex,yshift={#1}]
   \node (CQ) {\quotefont\fontsize{\quotesize}{\quotesize}\selectfont''};}

% select a colour for the shading
\colorlet{shadecolor}{lightgray}

\newcommand*\shadedauthorformat{\emph} % define format for the author argument

% Now a command to allow left, right and centre alignment of the author
\newcommand*\authoralign[1]{%
  \if#1l
    \def\authorfill{}\def\quotefill{\hfill}
  \else
    \if#1r
      \def\authorfill{\hfill}\def\quotefill{}
    \else
      \if#1c
        \gdef\authorfill{\hfill}\def\quotefill{\hfill}
      \else\typeout{Invalid option}
      \fi
    \fi
  \fi}
% wrap everything in its own environment which takes one argument (author) and one optional argument
% specifying the alignment [l, r or c]
%
\newenvironment{shadequote}[2][l]%
{\authoralign{#1}
\ifblank{#2}
   {\def\shadequoteauthor{}\def\yshift{-2ex}\def\quotefill{\hfill}}
   {\def\shadequoteauthor{\par\authorfill\shadedauthorformat{#2}}\def\yshift{2ex}}
\begin{snugshade}\begin{quote}\openquote}
{\shadequoteauthor\quotefill\closequote{\yshift}\end{quote}\end{snugshade}}



\title{Types and Programming Languages} 
\subtitle{Untyped Lambda Calculus}

\author{Rodrigo Bonif\'{a}cio}

\begin{document}

\setlength{\grammarparsep}{20pt plus 1pt minus 1pt} 
\setlength{\grammarindent}{4em}

\begin{frame}
\titlepage
\end{frame}

\begin{frame}
  \frametitle{$Lambda-Calculus$} 
\begin{shadequote}[l]{Benjamin Pierce}
  \ldots a formal system in which all computation
  is reduced to the basic operations of function definition
  and application. 
\end{shadequote}\pause

\begin{shadequote}[l]{Benjamin Pierce}
  \ldots Its importance arises from
  the fact that it can be viewed simultaneously as a {\color{blue}simple
  programming language and as a mathematical object} about
  which rigorous statements can be proved.
\end{shadequote}

\end{frame}

\begin{frame}

  \begin{shadequote}[l]{Benjamin Pierce}
    In the $lambda-calculus$ everything is a function: the arguments
    accepted by functions are themselves functions and the result returned
    by a function is another function. 
  \end{shadequote} \pause

  \begin{block}{BNF Grammar}
    \begin{grammar}
      <Exp> ::= x
      \alt      $\lambda$x.<Exp>
      \alt      <Exp> <Exp> 
    \end{grammar}  
  \end{block}
\end{frame}

\begin{frame}
  \frametitle{Syntax}

  Let $V$ be a countable set of variable names. The
  set of terms is the smallest set $T$ such that

  \begin{enumerate}
    \item $x \in T$ for every $x \in V$
    \item if $t_1 \in T$ and $x \in V$, then $\lambda x.t_1 \in T$
    \item if $t_1 \in T$ and $t_2 in T$, then $t_1\ t_2 \in T$   
  \end{enumerate}

  \pause
  The set of \emph{free variables} of a term $t$, written as
  $FV(t)$, is defined as follows:

  \begin{eqnarray*}
    FV(x) & = & {x} \\
    FV(\lambda x . t_1) & = &  FV(t_1) \backslash {x} \\
    FV(t_1\ t_2) & = & FV(t_1) \cup FV(t_2) 
  \end{eqnarray*}
  
\end{frame}

\begin{frame}
  \frametitle{Operational Semantics}

  \begin{shadequote}[l]{Benjamin Pierce}
    The computation consists of rewriting
    an application whose left-hand component is an
    abstraction, by {\color{blue}substituting} the right-hand
    component for the bound variable in the abstraction's body. 
  \end{shadequote} \pause

  \begin{block}{Graphically}
       \begin{prooftree}
       \AxiomC{$(\lambda x . e) e_2 \rightarrow [x \mapsto e_2]e$}
       \end{prooftree}
  \end{block}

  \pause
  \begin{itemize}
  \item where, $[x \mapsto e_2]e$ means ``the term obtained by
    replacing all \emph{free occurrences} of $x$ in $e$ by
    $e_2$''. 
  \end{itemize}
\end{frame}

\begin{frame}
       \begin{prooftree}
         \AxiomC{$t_1 \rightarrow t_1'$}
         \RightLabel{\quad (E-App1)}
         \UnaryInfC{$t_1\ t_2 \rightarrow t_1'\ t_2$}
       \end{prooftree}
       
       \begin{prooftree}
         \AxiomC{$t_2 \rightarrow t_2'$}
         \RightLabel{\quad (E-App2)}
         \UnaryInfC{$t_1\ t_2 \rightarrow t_1\ t_2'$}
       \end{prooftree}

       \begin{prooftree}
         \AxiomC{}
         \RightLabel{\quad (E-AppAbs)}
         \UnaryInfC{$(\lambda x . t_{12}) v_2 \rightarrow [x \mapsto v_2]t_{12}$}
       \end{prooftree}

\end{frame}

\begin{frame}[fragile]
\frametitle{Some examples of interesting functions}

 \begin{description}
    \item[Identity:] $(\lambda x . x)$ 
    \item[Function application:] $(\lambda f . \lambda a . (f a)$
    \item[Self application:] $(\lambda s . s s)$        
  \end{description}
  
\end{frame}

\begin{frame}
  However, let's follow a seamless
  transition from \emph{Untyped Arithmetic Expression}
  to the \emph{Untyped Lambda Calculus}\pause, discussing
  an intermediate (and more complex) language: \textsc{RBL-Calculus} 
\end{frame}

\begin{frame}
  \frametitle{\textsc{RBL-Calculus} syntax}

\begin{grammar}
  <Exp> ::= True
  \alt False
  \alt Num
  \alt Add <Exp> <Exp> \alt Sub <Exp> <Exp> 
  \alt And <Exp> <Exp> \alt Or <Exp> <Exp> \alt Not <Exp>
  \alt Let Var <Exp> <Exp>
  \alt Var 
  \alt $\lambda$ Var  <Exp>
  \alt <Exp> <Exp> 
\end{grammar}
\end{frame}

\begin{frame}[fragile]
  \frametitle{Some examples of programs in \textsc{RBL-Caulculus}}

\begin{verbatim}
  $ add 5 3 
  $ add 5 (5 3) 
  $ let x = 5 in x + x
  $ let inc = \x . x + 1 in inc 4   
\end{verbatim} 
\end{frame}

\begin{frame}
  \frametitle{Scope and Substitution}

  \texttt{let x = $e_1$ in $e_2$} \pause
  
  \texttt{let x = 5 in x + x} \pause 

  \texttt{let x = 5 in let x = 3 in x + 3} \pause  

  \texttt{let x = 5 in x + let y = x in x + y} 

\end{frame}

\begin{frame}
\huge{a bit of Haskell, \ldots + some test cases} 
\end{frame}

\begin{frame}[fragile]
  how can we translate programs in \textsc{RBL-Calculus} into
  \textsc{$\lambda-calculus$} programs?

  \pause

  \begin{block}{We will need a set of Haskell modules}
    \begin{description}
      \item[RBL:] the \textsc{RBL-Calculus} implementation
      \item[Lambda:] the \textsc{$\lambda-calculus$} implementation
      \item[Compiler:] the translator implementation
       \begin{verbatim}
          translate :: R.Expression -> L.Expression
       \end{verbatim}
    \end{description}
  \end{block}
\end{frame}

\begin{frame}
  Let's start with simple scenarios. First of all,
  \texttt{Let} expressions are redundant. That is,
  it is more or less easy to convert a \texttt{Let}
  expression into an application. 
\end{frame}

\begin{frame}
  Next, how can we represent an \textsc{If-Then-Else}
  statements using the \textsc{$\lambda-calculus}?$
\end{frame}

\end{document}
